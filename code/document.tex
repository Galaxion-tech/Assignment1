\let\negmedspace\undefined
\let\negthickspace\undefined
%\RequirePackage{amsmath}
\documentclass[journal,12pt,twocolumn]{IEEEtran}
%
% \usepackage{setspace}
 \usepackage{gensymb}
%\doublespacing
 \usepackage{polynom}
%\singlespacing
%\usepackage{silence}
%Disable all warnings issued by latex starting with "You have..."
%\usepackage{graphicx}
\usepackage{amssymb}
%\usepackage{relsize}
\usepackage[cmex10]{amsmath}
%\usepackage{amsthm}
%\interdisplaylinepenalty=2500
%\savesymbol{iint}
%\usepackage{txfonts}
%\restoresymbol{TXF}{iint}
%\usepackage{wasysym}
\usepackage{amsthm}
%\usepackage{pifont}
%\usepackage{iithtlc}
% \usepackage{mathrsfs}
% \usepackage{txfonts}
 \usepackage{stfloats}
% \usepackage{steinmetz}
 \usepackage{bm}
% \usepackage{cite}
% \usepackage{cases}
% \usepackage{subfig}
%\usepackage{xtab}
\usepackage{longtable}
%\usepackage{multirow}
%\usepackage{algorithm}
%\usepackage{algpseudocode}
\usepackage{enumitem}
 \usepackage{mathtools}
 \usepackage{tikz}
% \usepackage{circuitikz}
% \usepackage{verbatim}
%\usepackage{tfrupee}
\usepackage[breaklinks=true]{hyperref}
%\usepackage{stmaryrd}
%\usepackage{tkz-euclide} % loads  TikZ and tkz-base
%\usetkzobj{all}
\usepackage{listings}
    \usepackage{color}                                            %%
    \usepackage{array}                                            %%
    \usepackage{longtable}                                        %%
    \usepackage{calc}                                             %%
    \usepackage{multirow}                                         %%
    \usepackage{hhline}                                           %%
    \usepackage{ifthen}                                           %%
  %optionally (for landscape tables embedded in another document): %%
    \usepackage{lscape}     
% \usepackage{multicol}
% \usepackage{chngcntr}
%\usepackage{enumerate}

%\usepackage{wasysym}
%\newcounter{MYtempeqncnt}
\DeclareMathOperator*{\Res}{Res}
\DeclareMathOperator*{\equals}{=}
%\renewcommand{\baselinestretch}{2}
\renewcommand\thesection{\arabic{section}}
\renewcommand\thesubsection{\thesection.\arabic{subsection}}
\renewcommand\thesubsubsection{\thesubsection.\arabic{subsubsection}}

\renewcommand\thesectiondis{\arabic{section}}
\renewcommand\thesubsectiondis{\thesectiondis.\arabic{subsection}}
\renewcommand\thesubsubsectiondis{\thesubsectiondis.\arabic{subsubsection}}

% correct bad hyphenation here
\hyphenation{op-tical net-works semi-conduc-tor}
\def\inputGnumericTable{}                                 %%

\lstset{
%language=C,
frame=single, 
breaklines=true,
columns=fullflexible
}
%\lstset{
%language=tex,
%frame=single, 
%breaklines=true
%}
\begin{document}

%


\newtheorem{theorem}{Theorem}[section]
\newtheorem{problem}{Problem}
\newtheorem{proposition}{Proposition}[section]
\newtheorem{lemma}{Lemma}[section]
\newtheorem{corollary}[theorem]{Corollary}
\newtheorem{example}{Example}[section]
\newtheorem{definition}[problem]{Definition}
%\newtheorem{thm}{Theorem}[section] 
%\newtheorem{defn}[thm]{Definition}
%\newtheorem{algorithm}{Algorithm}[section]
%\newtheorem{cor}{Corollary}
\newcommand{\BEQA}{\begin{eqnarray}}
\newcommand{\EEQA}{\end{eqnarray}}
\newcommand{\define}{\stackrel{\triangle}{=}}
\newcommand*\circled[1]{\tikz[baseline=(char.base)]{
    \node[shape=circle,draw,inner sep=2pt] (char) {#1};}}
\bibliographystyle{IEEEtran}
%\bibliographystyle{ieeetr}
\providecommand{\mbf}{\mathbf}
\providecommand{\pr}[1]{\ensuremath{\Pr\left(#1\right)}}
\providecommand{\qfunc}[1]{\ensuremath{Q\left(#1\right)}}
\providecommand{\sbrak}[1]{\ensuremath{{}\left[#1\right]}}
\providecommand{\lsbrak}[1]{\ensuremath{{}\left[#1\right.}}
\providecommand{\rsbrak}[1]{\ensuremath{{}\left.#1\right]}}
\providecommand{\brak}[1]{\ensuremath{\left(#1\right)}}
\providecommand{\lbrak}[1]{\ensuremath{\left(#1\right.}}
\providecommand{\rbrak}[1]{\ensuremath{\left.#1\right)}}
\providecommand{\cbrak}[1]{\ensuremath{\left\{#1\right\}}}
\providecommand{\lcbrak}[1]{\ensuremath{\left\{#1\right.}}
\providecommand{\rcbrak}[1]{\ensuremath{\left.#1\right\}}}
\theoremstyle{remark}
\newtheorem{rem}{Remark}
\newcommand{\sgn}{\mathop{\mathrm{sgn}}}
\providecommand{\abs}[1]{\left\vert#1\right\vert}
\providecommand{\res}[1]{\Res\displaylimits_{#1}} 
\providecommand{\norm}[1]{\left\lVert#1\right\rVert}
%\providecommand{\norm}[1]{\lVert#1\rVert}
\providecommand{\mtx}[1]{\mathbf{#1}}
\providecommand{\mean}[1]{E\left[ #1 \right]}
\providecommand{\fourier}{\overset{\mathcal{F}}{ \rightleftharpoons}}
%\providecommand{\hilbert}{\overset{\mathcal{H}}{ \rightleftharpoons}}
\providecommand{\system}{\overset{\mathcal{H}}{ \longleftrightarrow}}
	%\newcommand{\solution}[2]{\textbf{Solution:}{#1}}
\newcommand{\solution}{\noindent \textbf{Solution: }}
\newcommand{\cosec}{\,\text{cosec}\,}
\providecommand{\dec}[2]{\ensuremath{\overset{#1}{\underset{#2}{\gtrless}}}}
\newcommand{\myvec}[1]{\ensuremath{\begin{pmatrix}#1\end{pmatrix}}}
\newcommand{\mydet}[1]{\ensuremath{\begin{vmatrix}#1\end{vmatrix}}}
\numberwithin{equation}{section}
\numberwithin{figure}{section}
\numberwithin{table}{section}
%\numberwithin{equation}{subsection}
%\numberwithin{problem}{section}
%\numberwithin{definition}{section}
\makeatletter
\@addtoreset{figure}{problem}
\makeatother
\let\StandardTheFigure\thefigure
\let\vec\mathbf
%\renewcommand{\thefigure}{\theproblem.\arabic{figure}}
\renewcommand{\thefigure}{\theproblem}
%\setlist[enumerate,1]{before=\renewcommand\theequation{\theenumi.\arabic{equation}}
%\counterwithin{equation}{enumi}
%\renewcommand{\theequation}{\arabic{subsection}.\arabic{equation}}
\def\putbox#1#2#3{\makebox[0in][l]{\makebox[#1][l]{}\raisebox{\baselineskip}[0in][0in]{\raisebox{#2}[0in][0in]{#3}}}}
     \def\rightbox#1{\makebox[0in][r]{#1}}
     \def\centbox#1{\makebox[0in]{#1}}
     \def\topbox#1{\raisebox{-\baselineskip}[0in][0in]{#1}}
     \def\midbox#1{\raisebox{-0.5\baselineskip}[0in][0in]{#1}}
\vspace{3cm}
\title{
	%\logo{
%Computational Approach to School Geometry
	ASSIGNMENT 1
%	}
}
\author{ CS21BTECH11020
}	


  \maketitle
  \bigskip
  \renewcommand{\thefigure}{\theenumi}
  \renewcommand{\thetable}{\theenumi}
  \numberwithin{equation}{section}
  \numberwithin{figure}{section}
  \numberwithin{table}{section}
  \section{Problem 6b (2018)}
  \begin{enumerate}[label=\thesection.\arabic*.,ref=\thesection.\theenumi]
  \numberwithin{equation}{enumi}
  \numberwithin{figure}{enumi}
  \numberwithin{table}{enumi} 
      \item If $\vec{A}=\myvec{
        2 & 3 \\
        5 & 7 \\
      }$,
      $\vec{B}=\myvec{ 
        0 & 4 \\
       -1 & 7 \\
      }$ and $\vec{C} =\myvec{ 
        1 & 0 \\
        -1 & 4 \\
       }$, 
       Find $\vec{A}\vec{C}+\vec{B^2}-10\vec{C}.$

  \solution We have,
    \begin{align}  
      \vec{A}&= \myvec{
        2 & 3 \\
        5 & 7 \\
      }\label{eq:1}
      \\ 
      \vec{B}&=\myvec{ 
        0 & 4 \\
        -1 & 7 \\
      }\label{eq:2}
      \\
      \vec{C}&= \myvec{
        1 & 0 \\
        -1 & 4 \\
      }\label{eq:3}
    \end{align}
    Since we know the identities ,
    \begin{align}
     \myvec{ 
        a & b \\
        c & d \\
      } 
     \myvec{ 
        w & x \\
        y & z \\
      } 
      &=
     \myvec{ 
        aw+by & ax+bz \\
        cw+dy & cx+dz \\
     } \label{eq:4}
    \\
     \myvec{ 
        a & b \\
        c & d
      } 
      \pm
     \myvec{ 
        w & x \\
        y & z
      } 
      &=
     \myvec{ 
        a\pm w & b\pm x \\
        c\pm y & d\pm z
      } \label{eq:5}
    \end{align}  
    Using Identity \eqref{eq:4}, we have
      \begin{align}
          \vec{A}\vec{C}&=
          \myvec{ 
              2 & 3 \\
              5 & 7
            } 
          \myvec{ 
              1 & 0 \\
              -1 & 4
          } 
          \\
          &=\myvec{ 
              2-3 & 0+12 \\
              5-7 & 0+28
          } 
          \\
          \vec{A}\vec{C}&=
          \myvec{ 
              -1 & 12 \\
              -2 & 28 
          }\label{eq:8} 
      \end{align}
      \begin{align}
          \vec{B^2}=\vec{B}\vec{B}
          &=
          \myvec{ 
              0 & 4 \\
              -1 & 7
          } 
          \myvec{ 
              0 & 4 \\
              -1 & 7
          }
          \\
          &=\myvec{ 
              0-4 & 0+28 \\
              0-7 & -4+49
          } 
          \\
          \vec{B^2}&=
          \myvec{ 
              -4 & 28 \\
              -7 & 45 
          }\label{eq:11}
    \end{align}
    and,
    \begin{align}
        10\vec{C}=10(\vec{I})\vec{C} &=
        \myvec{ 
            10 & 0 \\
            0 & 10
        } 
        \myvec{ 
            1 & 0 \\
            -1 & 4
        }
        \\
        &=\myvec{ 
            10+0 & 0+0 \\
            0-10 & 0+40
        }
        \\
        10\vec{C}&=
        \myvec{ 
            10 & 0 \\
            -10 & 40
        }\label{eq:14}
    \end{align}
    Using Identity \eqref{eq:5} and values from \eqref{eq:8}, \eqref{eq:11} and \eqref{eq:14}, We have\\
    $\vec{A}\vec{C}+\vec{B^2}-10\vec{C}=$
    \begin{align}
        & 
      \myvec{ 
          -1 & 12 \\
          -2 & 28
        } 
        +
      \myvec{ 
          -4 & 28 \\
          -7 & 45
        } 
        -
      \myvec{ 
          10 & 0 \\
          -10 & 40
      }
    \end{align}
    
    \begin{equation}
      \boxed{\vec{A}\vec{C}+\vec{B^2}-10\vec{C}=
       \myvec{ 
          \mathbf{-15} & \mathbf{40} \\
          \mathbf{1} & \mathbf{33}
        } 
      } 
    \end{equation}
  \end{enumerate}
\end{document}