\documentclass[journal,12pt,twocolumn]{IEEEtran}

%required pacakges
\usepackage{mathptmx} %bold mathematical text
\usepackage{amsmath,amssymb,amsthm} %math operation 
\usepackage[margin=0.6in]{geometry} %set margin
\usepackage{nopageno}   %remove page numbere

\title{\textbf{ASSIGNMENT 1}}
\author{\textbf{CS21BTECH11020}}
\date{}

\setlength{\parskip}{0.8em}
\parindent 0px

\begin{document}

  \maketitle
  %problem statement
  \textbf{PROBLEM 6b (2018) :} 
      If $ A= \begin{bmatrix}
        2 & 3 \\
        5 & 7 \\
      \end{bmatrix}$,
      $ B= \begin{bmatrix}
        0 & 4 \\
        -1 & 7
      \end{bmatrix}$ and 
      $ C = \begin{bmatrix}
        1 & 0 \\
        -1 & 4
      \end{bmatrix}$, find $ AC+B^2-10C$.

  \bigskip
  %solution
  \textbf{SOLUTION : }We have,
    $$ A= \begin{bmatrix}
      2 & 3 \\
      5 & 7 \\
    \end{bmatrix} \hspace{30pt}
     B= \begin{bmatrix}
      0 & 4 \\
      -1 & 7
    \end{bmatrix} \hspace{30pt}
     C = \begin{bmatrix}
      1 & 0 \\
      -1 & 4
    \end{bmatrix}$$
    
    %identites
    Since we know the identities,
    \begin{equation}
       \begin{bmatrix}
        a & b \\
        c & d
      \end{bmatrix}
      \begin{bmatrix}
        w & x \\
        y & z
      \end{bmatrix}
      =
      \begin{bmatrix}
        aw+by & ax+bz \\
        cw+dy & cx+dz
      \end{bmatrix}
    \end{equation}
    and,
    \begin{equation}
      \begin{bmatrix}
        a & b \\
        c & d
      \end{bmatrix}
      \pm
      \begin{bmatrix}
        w & x \\
        y & z
      \end{bmatrix}
      =
      \begin{bmatrix}
        a\pm w & b\pm x \\
        c\pm y & d\pm z
      \end{bmatrix}
    \end{equation}
    
    %calculating AC B^2 and 10C
    Using Identity $(1)$, we have

    $
      AC=
      \begin{bmatrix}
        2 & 3 \\
        5 & 7
      \end{bmatrix}
      \begin{bmatrix}
        1 & 0 \\
        -1 & 4
      \end{bmatrix}
      $

      $\implies AC=
      \begin{bmatrix}
        (2\times1)+(3\times(-1)) & (2\times0)+(3\times4) \\
        (5\times1)+(7\times(-1)) & (5\times0)+(7\times4)
      \end{bmatrix}
      $

      $\implies AC=
      \begin{bmatrix}
        2-3 & 0+12 \\
        5-7 & 0+28
      \end{bmatrix}
      $

      \begin{equation}
        \boxed{AC=
        \begin{bmatrix}
          -1 & 12 \\
          -2 & 28 
        \end{bmatrix}
        }
      \end{equation}

    
    $
      B^2=BB
      $

      $\implies B^2=
      \begin{bmatrix}
        0 & 4 \\
        -1 & 7
      \end{bmatrix}
      \begin{bmatrix}
        0 & 4 \\
        -1 & 7
      \end{bmatrix}
     $

      $\implies B^2=
      \begin{bmatrix}
        (0\times0)+(4\times(-1)) & (0\times4)+(4\times7) \\
        ((-1)\times0)+(7\times(-1)) & ((-1)\times4)+(7\times7)
      \end{bmatrix}
      $

      $\implies B^2=
      \begin{bmatrix}
        0-4 & 0+28 \\
        0-7 & -4+49
      \end{bmatrix}
      $

      \begin{equation}
      \boxed{B^2=
      \begin{bmatrix}
        -4 & 28 \\
        -7 & 45 
      \end{bmatrix}
      }
    \end{equation}
    
      $
      10C=(10I)C 
      $

      $\implies 10C=
      \begin{bmatrix}
        10 & 0 \\
        0 & 10
      \end{bmatrix}
      \begin{bmatrix}
        1 & 0 \\
        -1 & 4
      \end{bmatrix}
      $

      $\implies 10C=
      \begin{bmatrix}
        (10\times1)+(0\times(-1)) & (10\times0)+(0\times4) \\
        (0\times1)+(10\times(-1)) & (0\times0+10\times4)
      \end{bmatrix}
      $

      $\implies 10C=
      \begin{bmatrix}
        10+0 & 0+0 \\
        0-10 & 0+40
      \end{bmatrix}
      $

      \begin{equation}
      \boxed{10C=
      \begin{bmatrix}
        10 & 0 \\
        -10 & 40
      \end{bmatrix}
      }
    \end{equation}
  
    % calculating AC+B^2-10C
    Using Identity (2) and values from (3), (4) and (5), We have
    $$
      AC+B^2-10C= 
      \begin{bmatrix}
        -1 & 12 \\
        -2 & 28
      \end{bmatrix}
      +
      \begin{bmatrix}
        -4 & 28 \\
        -7 & 45
      \end{bmatrix}
      -
      \begin{bmatrix}
        10 & 0 \\
        -10 & 40
      \end{bmatrix}
    $$
    $\implies
      \begin{bmatrix}
        (-1)+(-4)-(10) & (12)+(28)-(0) \\
        (-2)+(-7)-(-10) & (28)+(45)-(40)
      \end{bmatrix}
    $
    
    %final result
    \begin{equation}
      \boxed{\mathbf{AC+B^2-10C} =
        \begin{bmatrix}
          \mathbf{-15} & \mathbf{40} \\
          \mathbf{1} & \mathbf{33}
        \end{bmatrix}
        }  
    \end{equation}

\end{document}
